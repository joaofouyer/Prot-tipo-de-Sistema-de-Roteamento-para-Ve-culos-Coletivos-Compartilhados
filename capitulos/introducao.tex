\chapter{Introdução} 
O tempo gasto no trânsito para a locomoção diária é um grande problema para os habitantes que residem em grandes metrópoles. Na região metropolitana de São Paulo, por exemplo, o congestionamento e o tempo gasto no trânsito é a quarto problema mais citado pelos cidadãos \cite{viveremsp}. Além disto, o transporte público se mostra ineficiente e causa desgaste para seus passageiros frequentes, sendo o quinto problema com o maior número de queixas \cite{mobilize}. Com pouca manutenção, muitas vezes os veículos da Grande São Paulo falham em oferecer uma boa experiência de locomoção para seus usuários, principalmente por causa da superlotação nos horários de pico e da falta de conforto dos veículos \cite{viveremsp}.

Além do problema enfrentado pelos passageiros, o Poder Público também tem o desafio de minimizar os gastos com a Secretaria de Transportes Metropolitanos ao passo que precisa diminuir o trânsito da cidade. Nos últimos anos, novas formas de locomoção foram incorporadas no cotidiano do paulistano a partir de iniciativas privadas ou de parcerias público-privadas (PPP) com ajuda da tecnologia digital, como por exemplo os aplicativos de carona, compartilhamento de bicicletas e patinetes elétricos e o fretamento privado de veículos coletivos -- como ônibus e vans. 

A evolução da tecnologia da informação, sobretudo da tecnologia móvel, desempenha um papel importante na viabilidade destes serviços, permitindo a interação entre o fornecedor e o consumidor mesmo que não se conheçam. Nos aplicativos de carona, por exemplo, passageiros e motoristas que nunca se viram conseguem ser conectados em tempo real graças à popularização dos \emph{smartphones} e da Internet móvel. 

Este trabalho apoia-se na capacidade que a tecnologia da informação tem para potencializar e facilitar o planejamento de rotas para veículos coletivos a partir de dados disponíveis.

\section{Motivação}

O congestionamento na locomoção urbana nas grandes metrópoles tornou-se um problema para seus habitantes e um desafio para gestores públicos. A duração total do deslocamento diário na cidade de São Paulo vem diminuindo nos últimos anos, junto com a utilização de transporte coletivo, reforçando a percepção de que medidas provisórias são adotadas como paliativo para problemas do transporte público que nunca são resolvidos.
\begin{quote}
Mesmo com sensível redução nos tempos de deslocamento dentro da cidade, as demandas por melhorias na lotação e no conforto nos tempos de espera e de viagem de transporte coletivo permanecem latentes, geram descontentamento e impactam diretamente a avaliação do sistema de transporte público da cidade e na disposição em utilizá-lo.
\end{quote}
\rightline{\cite[p. 36]{viveremsp}}

O transporte coletivo surge com diversos benefícios, como por exemplo uma alternativa que visa diminuir os problemas de congestionamento e os custos com combustível. Entretanto, alguns desafios dificultam ampla adoção de transporte coletivo, como o conforto que o veículo individual oferece ao acesso \emph{door-to-door} \cite{FURUHATA201328}, e a oferta e qualidade dos serviços no transporte público.

Um sistema de roteamento para o transporte coletivos de passageiros oferece a capacidade de minimizar a quantidade de assentos livres em um veículo com o objetivo de diminuir o número de veículos utilizados no transporte de passageiros e consequentemente o congestionamento nas grandes cidades. \cite{MOURAD2019323}

Paulistanos levam, em média, 1h57 no deslocamento diário para realizar a principal atividade do dia, como por exemplo o trabalho, escola ou faculdade. Considerando apenas passageiros de transporte público, este tempo aumenta para 2h08. Caso todas as atividades do dia sejam consideradas, a média da duração de deslocamento muda para 2h43 no geral, e 2h57 para usuários de transporte público \cite{viveremsp}. A atual configuração da mobilidade urbana também força passageiros a utilizarem mais de um modal em cada viagem, aumentando o cansaço dos seus usuários frequentes.

Outro desafio no planejamento urbano é atualizar as rotas e a frequência do transporte público conforme a demanda de um destino for alterado. Em média, o tempo de espera por um ônibus municipal na cidade de São Paulo é de 18 minutos \cite{viveremsp}, sendo a principal preocupação entre as pessoas que não utilizam o transporte, junto à superlotação e a alta tarifa pelo serviço prestado. A falta de planejamento em melhorar os problemas existentes no transporte coletivo, expõe em comparação à pesquisa dos anos anteriores, que não há mudanças significativas no transporte, configurando uma das causas dos altos custos operacionais e mantendo o descaso com seus usuários. \cite{viveremsp}.

A evolução da tecnologia da informação, sobretudo da tecnologia móvel e da Internet das Coisas \emph{(IoT)} possibilita que ferramentas auxiliares sejam utilizadas para coletar dados e realizar um planejamento mais adequado para transporte coletivo. Desta forma, rotas ociosas poderiam ser alteradas para diminuir a duração total do trajeto e diminuir os custos de operação, assim como novas rotas poderiam ser criadas para atender demandas existentes e melhorar a abrangência do serviço ofertado. Com ajuda das soluções tecnológicas já existentes como \emph{Google Maps} e \emph{Waze} é possível ampliar as funcionalidades que um simples \emph{GPS} pode proporcionar, fazendo com que a rota seja calculada para atender não só uma pessoa individualmente, mas várias delas ao mesmo tempo, levando em consideração o deslocamento e tempo de viagem.

A tecnologia da informação e métodos computacionais podem servir de apoio para ferramentas que visam facilitar e ampliar o planejamento das rotas de transporte coletivo de pessoas em uma metrópole como São Paulo, diminuindo a dificuldade de gestores de transportes públicos e de administradores de empresas de fretado em planejar essas rotas para os veículos que as realizarão. Neste trabalho, será desenvolvido um protótipo que aplica e adapta conjuntos de métodos computacionais para aprimorar o roteamento de veículos para o transporte de passageiros, seguindo critérios e restrições que serão detalhadas nos objetivos e delimitação do problema.

\section{Objetivos}

Problemas de roteamento de veículos em geral são multiobjetivo \cite[p. 279]{goldbarg}, onde é possível resolver dois ou mais objetivos simultaneamente, aplicando algumas restrições\cite[p. 2]{multi-objetivo}. Neste trabalho, a distância percorrida e o número de assentos livres no veículo são os objetivos a serem abordados, porém será necessário impor restrições para que não sejam conflitantes.
Nas subseções a seguir, serão apresentados o objetivo geral, os objetivos específicos e as restrições deste trabalho. 

\subsection{Objetivo Geral}

Desenvolver um protótipo de roteamento de veículos para o transporte de passageiros em veículos compartilhados, objetivando minimizar o custo da atividade. O custo da atividade é determinado pela somatória das estimativas de distância e de tempo associados ao percurso. 

\subsection{Objetivos Específicos}
Nos objetivos específicos deste trabalho pretende-se:
\begin{itemize}
    
    \item Agrupar passageiros que possam convergir em uma rota respeitando a capacidade do veículo;
    
    \item Maximizar o número de passageiros em uma mesma rota;
    
    \item Determinar pontos de embarque e desembarque dos passageiros;
    
    \item Determinar horários de embarque e desembarque dos passageiros;
    
    \item Atualizar a rota de acordo com as solicitações de inclusão ou cancelamento.
    
\end{itemize}

\subsection{Restrições}
As condições restritivas deste trabalho envolvem as decisões táticas escolhidas para o sistema de roteamento de veículos. Sendo elas:
\begin{itemize}
    \item Localizar a área delimitadora para o embarque do passageiro;
    
    \item Localizar a área delimitadora para o desembarque do pasageiro;
    
    \item Verificar a capacidade do veículo para todos os passageiros viajarem sentados;
    
    \item Definir o número de veículos que irá atender a linha criada;
    
\end{itemize}

\section{Justificativa}
A partir da Emenda Constitucional número 90 de 2015 \cite{pec}, o transporte é considerado um direito social de todo cidadão brasileiro, junto com educação, saúde e previdência social. Apesar disto, o sistema de transporte público ainda é deficiente em diversos municípios brasileiros, mesmo apresentando despesas altas aos cofres públicos e nas tarifas pagas pelos seus usuários. 
 
No caso do município de São Paulo, o valor orçado para a secretaria de transportes urbanos em 2019 foi de aproximadamente R\$ 3,4 bilhões \cite{loatrans}. O alto investimento não é justificado em melhoria na qualidade do serviço, mas pode ser atribuído a diversos fatores, como a falta de trabalhadores qualificados e de recursos necessários para planejar e gerenciar a rede de transporte, que inclui estabelecer o itinerário e horário das rotas de ônibus de acordo com a demanda, designar e atribuir os veículos às rotas, considerando o custo, a segurança, o tempo de deslocamento e a capacidade dos veículos \cite{lima2016mixed}.

Por outro lado, ambos Governo do Estado e da Cidade de São Paulo, facilitam a coleta e contextualização de informações referentes ao transporte e locomoção na região metropolitana de São Paulo por meio de pesquisas realizadas e de ferramentas disponíveis para demais pesquisadores. São exemplos a pesquisa de Origem e Destino realizada pelo Metrô de São Paulo (METRÔ, \citeyear{odpq}), a interface para acesso à informação para desenvolvedores fornecida pela SPTrans \cite{sptranspq} e a composição do estudo tarifário que é realizado anualmente \cite{tarifapq}. Além disto, em face dos desafios de mobilidade urbana e do esforço coletivo em superá-los, o Governo do Estado criou um programa exclusivo de Parceria Público-Privada (PPP) para mobilidade na região metropolitana (GESP, \citeyear{ppppq}).

Este esforço coletivo também é evidente em organizações não-governamentais e em setores privados. A Rede Nossa São Paulo, por exemplo, possui pesquisas que tangem o assunto sob diferentes aspectos \cite{redepq}. São Paulo é a única cidade brasileira no Movement \cite{uberpq}, plataforma da Uber que disponibiliza para gestores públicos e pesquisadores informações detalhadas sobre o tempo de locomoção de acordo com as viagens realizadas no serviço.

Portanto, o problema que emerge deste trabalho é de como a tecnologia da informação pode servir de ferramenta para auxiliar no planejamento de rotas para veículos coletivos. O instrumento resultante poderia ser utilizado por gestores públicos e/ou pela iniciativa privada ao desenvolver formas alternativas de transporte coletivo. A escolha de São Paulo se deve, principalmente, à disponibilidade de dados e de pesquisas que podem ser utilizadas no desenvolvimento do trabalho.


\section{Delimitação do Problema}
O estudo do modelo de solução do problema considera apenas a região da Grande São Paulo, devido à disponibilidade de dados de pesquisas anteriores e de ferramentas do setor público e privado. São ignoradas tanto restrições físicas -- como a inviabilidade de um veículo passar por uma via estreita -- quanto legais -- como a Zona de Máxima Restrição de Fretamento (ZMRF) ou licenciamento necessário para um veículo realizar o transporte coletivo de pessoas. O processamento do georreferenciamento do mapa, pontos e das vias será realizado com suporte das bibliotecas \emph{Places API}, \emph{Directions API}, \emph{Distance Matrix API}, \emph{Geocoding API} e \emph{Maps Embed API}, o uso é detalhado na fundamentação teórica.

Visando o conforto em realizar uma viagem inteira sentado, a quantidade de passageiros que podem ser transportados em um veículo será considerada a partir da quantidade de assentos disponíveis. Será levado em consideração que uma viagem deverá ser realizada a partir da origem especificada pelo usuário -- normalmente a residência -- até o destino -- colégio, faculdade ou trabalho -- desconsiderando possíveis integrações com outros modais de transporte -- ônibus municipais, trens ou metrôs -- ou outros fretados. Caso algum usuário ache conveniente realizar alguma integração, ele deverá explicitamente definir o endereço de onde a integração é realizada como origem e/ou de destino.

\section{Método de Trabalho}\label{metodo-trabalho}
A pesquisa irá se basear em procedimentos que consistem em estudos bibliográficos que envolve a leitura de livros, artigos acadêmicos, notícias e informações do poder público e organizações não-governamentais para planejamento de mobilidade urbana. A pesquisa também se baseará em estudos de casos, visto que abordará empresas de transportes públicos e privados e modelos adotados em outras cidades. Por fim, será realizada a modelagem e a implementação do protótipo e análise dos resultados.

\begin{enumerate}
\item \textbf{Estruturação do Problema} --
Contextualizar e entender os desafios da mobilidade urbana e as soluções que já são utilizadas pela engenharia de tráfego para contorná-los. Grandes metrópoles como São Paulo contam com inteligência de tráfego para entender melhor problemas de trânsito, demandas e as rotas de transportes coletivos e soluções mais robustas para cada um dos problemas citados. Diversos trabalhos acadêmicos publicados também propõem soluções diversas para melhorar o transporte nas grandes cidades em contextos diferentes.


\item \textbf{Revisão Bibliográfica} --
Estudar métodos para soluções de problemas de mobilidade urbana, logística, heurística e meta-heurística que já foram publicados e contextualizá-los com o problema deste trabalho. Entender como a sociedade urbana se organiza em rede e como a topologia pode ser aplicada na locomoção diária de uma metrópole como São Paulo.  A ideia de rede pode ser analisada a partir da metodologia de agregar insumos de uma mesma natureza ou de naturezas diferentes. ``A rede é o próprio tecido constitutivo da sociedade, engendrando desde as redes sociais pessoais'' \cite{inojosa}. Analisar as diferentes formas de armazenar e indexar pontos e rotas georreferenciadas.

\item \textbf{Definição e implementação do protótipo} --
Modelar os casos de uso, as variáveis de decisão, as restrições e modelo de relação entre veículos, rotas, motoristas, passageiros e pontos. Definir tecnologias que serão utilizadas e do método de desenvolvimento do protótipo.

\item \textbf{Planejamento e execução de testes} --
Criar a massa de testes para medir a eficiência do protótipo com simulações e interpretar seus resultados.

\item \textbf{Análise de resultados} -- 
Analisar os resultados fornecidos pelos testes e verificar se cumprem os requisitos do protótipo.
\end{enumerate}

\section{Organização do Texto}

O projeto é apresentado em 5 capítulos, referências bibliográficas e os anexos.

No capítulo 1, Introdução, são apresentados  a problemática, motivação, objetivos gerais e específicos e a metodologia de trabalho que será utilizada neste projeto.

No capítulo 2, Revisão Bibliográfica, serão apresentados a Fundamentação Teórica que abordará a complexidade dos problemas de roteamento de veículos, seja no contexto de logística para entregas ou para transporte de passageiros e os trabalhos relacionados ao tema proposto. 

No capítulo 3, Desenvolvimento, serão apresentados o trabalho de pesquisa e o processo de desenvolvimento do protótipo.

No capítulo 4, Resultados Obtidos, serão apresentados os testes para aplicação no protótipo e seus resultados, a fim de interpretá-los  

Por fim, no capítulo 5, Conclusão, será exposta a análise crítica do problema, os objetivos atingidos e quais são os pontos de melhorias para trabalhos futuros.