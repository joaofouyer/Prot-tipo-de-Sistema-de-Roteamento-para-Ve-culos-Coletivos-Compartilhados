\chapter{Revisão Bibliográfica}

Neste capítulo é apresentada a revisão da bibliografia que dará suporte e embasamento para o desenvolvimento deste trabalho, expondo técnicas, conceitos fundamentais e tecnologias que são citadas na literatura para a resolução do problema proposto. Os principais fundamentos teóricos que são cruciais para a compreensão do trabalho são abordados na seção \ref{fundamentacao}. A proposta para o método de desenvolvimento do protótipo é discutida na seção \ref{metodo-desenvolvimento}. Por fim, alguns trabalhos relacionados são explorados na seção \ref{relacionados}, onde são destacadas as semelhanças, diferenças e contribuições para o trabalho.

\section{Fundamentação Teórica}\label{fundamentacao}
Nesta seção são apresentados conceitos fundamentais para a compreensão do presente trabalho. Na seção \ref{transporte-urbano}, é apresentada uma contextualização sobre o transporte de passageiros em veículos coletivos urbanos. Na seção \ref{problemas-de-otimizacao} são apresentados problemas de otimização, incluindo um problema clássico de otimização denominado \emph{Vehicle Routing Problem} e suas classes, que são detalhadas na subseção \ref{vrp}. 

As principais técnicas computacionais para a resolução de problemas de Otimização Combinatória, incluindo o Problema de Roteamento de Veículos são introduzidas na subseção \ref{tecnicas-computacionais} exploradas em \ref{gulosos} (Algoritmos Gulosos) \ref{metodos-exatos} (Métodos Exatos), \ref{heuristica} (Métodos Heurísticos) e \ref{metaheuristica} (Métodos Meta-Heurísticos). Por fim, sistemas de georreferenciamento citados na literatura são apresentados na subseção \ref{georreferenciamento}

\subsection{Transporte Coletivo Urbano}\label{transporte-urbano}
Existem diversos atributos na relação do passageiro com o transporte público coletivo, mas esses atributos são independentes e devem ser estudados de forma isolada. A resolução do Contran \cite{denatran} é feita para que o veículo fique de acordo com cada atributo, sendo eles:
\begin{itemize}
    \item confiabilidade;
    \item tempo;
    \item acessibilidade;
    \item conforto;
    \item segurança;
    \item custo (tarifas).
\end{itemize}

No \nameref{anexo-tabela-indicadores}, consta a tabela \ref{cotran-table} que 
relaciona o atributo do serviço com a sua definição e um indicador equivalente para a sua avaliação.
 
Os atributos de tempo e capacidade dos veículos caracterizam o transporte urbano coletivo e estão ligados no projeto do veículo, que são prefixados pelos fabricantes. Para o passageiro,o aspecto conforto é composto por um veículo com pouca ocupação, garantindo que o passageiro estará sentado durante a viagem, mas o que também faz parte da satisfação do usuário é a extensão do deslocamento.

A Figura \ref{figura-ocupacao-tempo} a seguir representa a relação entre a densidade de ocupação de um veículo (em pessoas por $m^2$) e o tempo de deslocamento (em minutos).

\begin{figure}[H]
  \centering
  \caption{Relação entre densidade de ocupação e tempo de deslocamento}
 \includegraphics[scale=0.55]{imagens/limite-fisiologico.png} \par
\bigskip
Fonte: \cite[p.307]{MARTINS}
\label{figura-ocupacao-tempo}
\end{figure}
Assim, em relação ao transporte de passageiros, os princípios que devem estar presentes para um bom nível de serviço, em destaque, são: 
\begin{itemize}
    \item \textbf{Permanência}: um serviço constante, sem problemas de prestação de serviços.
    \item \textbf{Eficiência}: oferta de um serviço satisfatório (conforto).
    \item \textbf{Modicidade}: serviços a preços razoáveis
\end{itemize}

Os atributos mencionados serão auxiliarão a definir as restrições na hora de calcular a rota e para definir os objetivos que precisam ser atingidos.

\begin{itemize}
\item A permanência garante que o serviço prestado será do começo ao fim, sem integração de modais e com horários pré-estabelecidos com o usuário;
\item A eficiência visa que o usuário terá seu assento reservado e uma previsão de tempo de chegada definida;
\item A modicidade trará o transporte coletivo como o meio de dividir o valor dos custos, como manutenção, entre todos os passageiros e assim, o preço será reduzido em relação ao valor do uso de veículo particular.
\end{itemize}



\subsection{Problemas de Otimização}\label{problemas-de-otimizacao}

Problemas de otimização podem ser fundamentalmente divididos entre aqueles com variáveis contínuas ou de natureza discreta \cite{combinatorial-optimization}. Nos problemas de otimização de variáveis contínuas procura-se por uma função ou um conjunto de números \(R \in \textrm I\!R \) para a solução.

Já os problemas de otimização de natureza discreta podem ser solucionados por meio de técnicas de Otimização Combinatória. Nos Problemas de Otimização Combinatória procura-se um conjunto finito de soluções -- naturalmente inteiras -- e que podem ser formulados como Problemas de Programação Inteira e consequentemente solucionados por meio de Algoritmos de Programação Inteira \cite{goldbarg} ou podem ser formulados por meio de análise combinatória e solucionados como problemas em Grafos ou algoritmos especificamente desenvolvidos.


Problemas de Otimização Combinatória podem ser formulados como \((F, c)\) onde \(F\) é um conjunto de pontos possíveis; e \(c\) uma função de custo \cite{rodolfo}.

\[c: F \longrightarrow R \]

Onde o problema é encontrar um\(f \in F\) para qual

\[c(f) \leq c(y): y \in F \]

O ponto \(f\) é denominado de ótimo global ou solução ótima.

O Problema de Roteamento de Veículos (\emph{Vehicle Routing Problem} - VRP) é um problema clássico de otimização combinatória e que será detalhado na subseção a seguir.

\subsubsection{Vehicle Routing Problem (VRP)} \label{vrp}

O Problema de Roteamento de Veículos (VRP) foi introduzido na literatura em 1959 por Dantzig e Ramser \cite{toth}. O VRP tem sido, particularmente, um dos problemas mais abordados nas áreas de Otimização Combinatória e Pesquisa Operacional nas últimas décadas \cite{steiner2000problema}. E desde então, vários cenários foram derivados, como o roteamento para veículos de entrega, roteamento de veículos escolares ou o roteamento de veículos para transporte de passageiros no contexto urbano, que é o objeto de estudo do trabalho presente. De modo geral, consiste em definir rotas para cumprimento de demandas, considerando o número de veículos disponíveis e respeitando a capacidade de cada veículo.

O Problema de Roteamento de Veículos Abertos (\emph{Open Veihicle Routing Problem} -- \emph{OVRP}) é uma das existentes derivações do \emph{VRP}. Uma das soluções do OVRP consiste em ser resolvida por meio de um grafo, sendo ele definido por G = (V,E) onde V é um conjunto finito cujos elementos são denominados vértices e E é um conjunto de subconjuntos de dois elementos de V denominados arestas \cite{nogueira2015introduccao}. Desta forma, o \emph{OVRP} consiste em caminhos hamiltonianos, isto é, apresenta um ciclo que percorre todos os vértices do grafo, para um conjunto de clientes atribuídos num mesmo veículo a partir de uma demanda total de todos os clientes de uma rota e que não deve exceder capacidade do veículo.  Cada cliente é visitado apenas uma vez por um dos veículos. Porém, o \emph{OVRP} é dado como problema de coleta e entrega de carga e não propriamente de passageiros \cite{BRANDAO2004552}.

No problema em questão, a partir de passageiros coincidentes de uma região, o objetivo é transportá-los para um local próximo ao seu destino, minimizando custos e melhorando frotas para atendimento de regiões condizentes. 

Tendo em vista ser um problema de otimização e combinatória, há diversas abordagens para a sua solução, seja com algoritmos gulosos ou por soluções exatas, heurísticas e metaheurísticas. Primeiro será realizada uma descrição das diferentes abordagens, seus prós e contras e depois os principais algoritmos citados na literatura em cada uma delas, de forma que seja possível decidir qual deles é mais adequado de acordo com a solução para o seu problema e o candidato para aplicar no protótipo.

Por fim, sabe-se que no contexto do problema de roteamento de veículos, muitas vezes ainda não existem técnicas computacionais capazes de garantir que o caminho encontrado será o menor ou o mais rápido em um tempo polinomial, classificando este tipo de problema como NP-Difícil. Desta forma, a solução encontrada pode não ser ótima global, mas sim a local.

\subsection{O Problema do Caixeiro Viajante}

O problema do Caixeiro Viajante (\emph{Travelling Salesman/Salesperson Problem} -- \emph{TSP}) é um problema clássico de otimização e combinatória e está entre os antigos problemas já estudados na área. Foi formulado matematicamente pela primeira vez em 1932, por Karl Menger \cite{oliveira2015extensoes} e possui muitas variações.
Este problema pode ser representado por grafo. Para o TSP as cidades são os vértices e as ligações entre as cidades são as arestas do grafo. Para formulação do problema, dado um conjunto de cidades é \emph{``como o caixeiro passaria por todas as cidades, uma única vez com o menor trajeto possível?''} A resposta dessa questão, com o caixeiro retornando para a cidade de origem, gera uma ordenação das cidades que o caixeiro deve percorrer. Sua variante mais conhecida é a descrita acima, também conhecido como \emph{TSP com retorno}. Caso não faça parte da solução retornar para a cidade inicial, a variante é conhecida como \emph{TSP sem retorno}, sendo a variante que será utilizada neste trabalho. 


\subsection{Técnicas Computacionais para Resolução de Problemas de Otimização Combinatória}
\label{tecnicas-computacionais}
O desenvolvimento da capacidade de processamento dos computadores nos últimos anos tem servido de suporte para a solução -- mesmo que parcial -- de problemas de Otimização Combinatória, algumas vezes envolvendo centena de milhares de variáveis e restrições.

Diferentes soluções podem ser utilizadas para o VRP.  Fundamentalmente a literatura os separa entre três principais abordagens: Métodos Exatos, Métodos Heurísticos e Meta-Heurísticos \cite{maxwell}. \begin{comment}
Antes mesmo de abordar os métodos de soluções do VRP, é necessário entender as classes que os algoritmos de roteamento podem pertencer e, assim, será possível escolher os melhores candidatos para a solução do problema proposto.
\end{comment} 
Nas subseções seguintes, um levantamento dos métodos mais frequentes para a solução foi realizado a fim de guiar na decisão da escolha da abordagem para resolver o problema proposto pelo trabalho.

\subsection{Algoritmos Gulosos}
\label{gulosos}
Algoritmos com estratégia gulosa (=gananciosa =\emph{greedy}) \cite{feofiloff} são os relacionados com problemas de otimização que não repensam a solução tomada, verificando a informação disponível no momento sem se preocupar com efeitos futuros. Quando um algoritmo guloso se baseia no princípio da otimalidade e funciona, a primeira solução encontrada será sempre a solução ótima, um exemplo seria um algoritmo para achar o caminho mais curto entre dois pontos. Se o problema é considerado difícil, é comum que a exigência de encontrar a solução ótima seja descartado, mas se mantenha a busca pela solução que possa obter resultado mais próximo da solução ótima \cite[p. 60]{ziviani}.

\begin{quote}
Não existe necessidade de avaliar alternativas, nem de empregar algoritmos sofisticados que permitam desfazer decisões tomadas previamente. A razão de o algoritmo ser chamado guloso é que o algoritmo escolhe, a cada passo, o candidato mais evidente que possa ser adicionado à solução.
\end{quote}
\rightline{\cite[p. 58]{ziviani}}

\subsection{Métodos Exatos} \label{metodos-exatos}

Em métodos exatos uma varredura é realizada sobre todo o domínio de busca das soluções, garantindo que o ótimo global seja encontrado sempre que a solução for viável. Entretanto, algoritmos de métodos exatos não podem ser executados em tempo polinomial, de forma que a equação de tempo em função dos número de insumos cresce de forma polinomial \cite{maxwell}. 

São apresentados alguns métodos exatos mais citados pela literatura para resolver o VRP no \nameref{anexo-metodos-exatos}.

%βρίσκω%
%$\varepsilon \upsilon \rho \iota \sigma \kappa \omega$
\subsection{Métodos heurísticos}\label{heuristica}
A etimologia da palavra \emph{heurística} tem origem do verbo grego $\beta \rho \iota \sigma \kappa \omega$
(\emph{evrísko}) -- cuja conjugação na primeira pessoa é  $\varepsilon \upsilon \rho 	\eta \kappa \alpha$ (\emph{heúrēka}) -- que significa descobrir. Quando a palavra é empregada no contexto de métodos para a solução de problemas de Otimização Combinatória, o seu sentido pode ser ampliado conforme, por exemplo, a definição de Kendra van Wagner-Cherry \emph{``Uma heurística é um atalho mental que permite às pessoas resolver problemas e fazer julgamentos rápida e eficientemente.''} \cite{goldbarg}


Métodos heurísticos são aqueles que garantem um bom tempo computacional, porém, não garantem uma solução ótima em todos os casos que atuam. No caso de uma aplicação, uma aproximação heurística traz o benefício do tempo de processamento, uma vez que uma solução ótima, com alto tempo de processamento, deixa de ser uma boa solução para o usuário. 

São apresentados alguns métodos heurísticos mais citados pela literatura para resolver o VRP no \nameref{anexo-metodos-heuristicos}.

\subsection{Métodos meta-heurísticos}\label{metaheuristica}
O prefixo \emph{meta} é de origem grega e seu significado original é \emph{além}. Entretanto, em diversas áreas de conhecimento, o seu uso normalmente é empregado para fornecer um significado sobre algo de sua própria natureza, como um conceito que é uma abstração que explica um outro conceito. Desta forma, \emph{meta-heurísticas} são heurísticas projetadas para encontrar, gerar, ou selecionar heurísticas.

Formalmente, uma meta-heurística é composta por um conjunto de regras estruturadas que podem servir de base para a projeção da solução a partir de uma gama de heurísticas computacionais. Estratégias genéricas derivadas de heurísticas  podem ser aplicadas em variados contextos de problemas de otimização a partir de adaptações para resolver um problema específico \cite{maxwell}. Muitas meta-heurísticas são inspiradas em fenômenos naturais, como a Colônia de Formigas ou Enxame de Abelhas. Assim como na heurística, a solução ótima não é garantida nos métodos meta-heurísticos. A definição de Goldbarg \emph{et al.} (2015) indica o agrupamento de heurísticas computacionais para compor um sistema geral de regras: 

\begin{quote}
Trata-se de uma arquitetura geral de regras que, formada a partir de um tema em comum, pode servir de base para o projeto de uma ampla gama de heurísticas computacionais.
\end{quote}
\rightline{\cite[p. 75]{goldbarg}}

A Figura \ref{heuristica-goldbarg} exibe características genéricas de uma abordagem meta-heurística (indicado na figura como Regras Gerais) para agrupar e adaptar heurísticas de determinado tema em comum para a resolução de mais de um problema dentro do contexto dos Problemas de Aplicação.

\begin{figure}[H]
  \centering
  \caption{Arquitetura geral para meta-heurísticas}
 \includegraphics[scale=0.4]{imagens/metaheuristica.png} \par
\bigskip
\label{heuristica-goldbarg}
    Fonte: \cite[p. 75]{goldbarg}
\end{figure}

\subsubsection{Métodos meta-heurísticos citados pela literatura no VRP} \par
Na Tabela \ref{tabu-relacionados} são listadas algumas referências de trabalhos acadêmicos relacionados que já utilizaram métodos meta-heurísticos para encontrar a solução.

\begin{table}[ht]
\centering
\resizebox{\textwidth}{!}{%
\begin{tabular}{ccc}
\hline
\textbf{Referência} & \textbf{Tema} \\ \hline
\cite{TAILLARD1991443} & \begin{tabular}[c]{@{}c@{}}Robust taboo search for the quadratic\\ assignment problem\end{tabular} \\ \hline
\cite{TSUBAKITANI1998113} & An empirical study of a new metaheuristic for the traveling salesman problem \\ \hline
\cite{eglese} & Open vehicle routing problem \\ \hline
\end{tabular}%
}
\caption{Trabalhos relacionados que utilizaram práticas da busca tabu.}
\label{tabu-relacionados}
\end{table}

São detalhados alguns destes métodos meta-heurísticos mais citados pela literatura para resolver o VRP no \nameref{anexo-metodos-metaheuristicos}.

% \subsubsection{Busca em uma Vizinhança Grande Adaptada}

\subsection{Sistemas de Georreferenciamento}\label{georreferenciamento}

Georreferenciamento é a prática de obter determinado local (ou ponto) no mapa e inserir suas coordenadas em um sistema de referências planas. No contexto de Georreferenciamento, os pontos escolhidos para a extração das coordenadas são chamados de Pontos de Controle. As coordenadas reais dos pontos de controle (verdade terrestre) podem ser obtidas em trabalho de Campo (Lev. Topográfico, GPS, mapas em papel ou digitais georreferenciados) \cite{georreferenciamento}.

O \emph{Google} fornece uma vasta gama de ferramentas de geolocalização que contribuem para o desenvolvimento de vários aplicativos e softwares com o objetivo guiar o usuário de alguma forma a um destino. Dentre elas, temos algumas \emph{APIs} que se destacam quando se tem como objetivo criar uma plataforma de roteamento veicular, sendo elas:
\begin{itemize}
    \item \textbf{Google Places API} -- É um serviço que fornece informações sobre um lugar a partir de requisições HTTP,  os lugares são definidos nessa API como estabelecimentos, localizações geográficas ou pontos de interesse proeminentes;
    \item \textbf{Google Directions API} -- Fornece rotas de transporte público, bicicleta, carro e a pé. Calcula os tempos de deslocamento atuais ou futuros com base no trânsito em tempo real;
    \item \textbf{Google Distance Matrix API} -- Fornece tempos de deslocamento e distâncias para um ou mais locais.;
    \item \textbf{Google Geocoding API} -- Transforma dados de endereço de ruas em dados de altitude e latitude;
    \item \textbf{Google Maps Embed API} -- Realiza a inclusão de uma interface gráfica de mapeamento da própria Google.
\end{itemize}

Esses serviços terão um papel importante no desenvolvimento da plataforma de roteamento de transportes coletivos. O uso dessas ferramentas facilita a entrada de dados na plataforma e gera mais consistência no desenvolvimento deste projeto.


\begin{comment}

\subsection{Agrupamento de Passageiros Candidatos à Rota Através de Grafos de Similaridade e Spectral Clustering}

Agrupamento (\emph{Clustering}) é uma técnica muito usada hoje em dia, em qualquer área que lida com dados empíricos (valores numéricos concretos). Trabalhadores e pesquisadores que têm essa quantidade de dados em mão, tentam agrupar os dados que tem certo grau de similaridade com o objetivo de manter suas informações organizadas e facilitar o manuseio desses dados. Uma variação desta técnica cujo a performance toma destaque sobre os algoritmos de agrupamento padrão é o \emph{Spectral Clustering}. O \emph{Spectral Clustering}  agrupa nós de uma matriz de adjacências (ou grafo de similaridades) e dados considerados similares através do uso de métodos de álgebra linear e cálculo de grafos Laplacianos \cite{von2007tutorial}.

O algoritmo de \emph{Spectral Clustering} é dividido nos seguintes passos \cite{shi2000normalized}:

\begin{enumerate}
    \item Entrada: Uma matriz de similaridade $S \in \mathbb{R}^{n*n}$, e o número \emph{k} de grupos a serem construídos;
    \item Construção do grafo de similaridade a partir de pontos \(X_1,...,X_n\) com noções de similaridade \(S_{ij} >= 0\)  entre todos os pontos \(X_i e X_j\).
    \item Computar o Laplaciano \emph{L} não-normalizado.
    \item Computar os primeiros \emph{k} eigenvetores $U_1,...,U_k$ da generalização do eingenproblema $Lu =  \lambda Du$.
    \item Seja  $U \in  \mathbb{R}^{n*k}$ a matriz contendo os vetores $U_1,...,U_k$ sendo as colunas.
    \item É feito um \emph{loop} (for) de \emph{1} até \emph{n}  onde $yi \in  \mathbb{R}^k$ é o vetor correspondente à \emph{i-ésima} linha das linhas de \emph{U}.
    \item Os pontos de agrupamento $(y_i)_{i=1,...,n}$ em $ \mathbb{R}^k$ com os algoritmos de \emph{k-means} (No \nameref{anexo-k-nearest} está o algoritmo \emph{KNN} que é utilizado para este tipo de problema) em grupos de $C_1,...,C_k$.
    \item Saída: Grupos $A_1,...,A_k$ com $A_i = {j |yj \in C_i}$.
\end{enumerate}
\end{comment}

\section{Método de Desenvolvimento}
\label{metodo-desenvolvimento}
Nesta seção, é detalhado o método de desenvolvimento para a implementação da solução e do protótipo seguindo conceitos de engenharia de software e modelagem.

\subsection{Iconix}
O ICONIX \cite{iconixuml} é um método de desenvolver software orientado a objetos dentre os métodos ágeis composta pelas atividades Análise de Requisitos, Análise Preliminar, Projeto e Implementação. É um método simplificado que divide todas as fases de modelagem em duas visões iterativas: dinâmica e estática.

A Figura \ref{iconix-img}   é o resumo de como funciona o processo ICONIX. No processo, são utilizados os diagramas de Caso de Uso, Robustez e de Sequência na fase da visão dinâmica da modelagem e na fase da visão estática os diagramas de Domínio e de Classes.

\begin{figure}[H]
  \centering
  \caption{Representação do processo ICONIX}
 \includegraphics[scale=0.7]{imagens/iconix.jpg} \par
\bigskip
\label{iconix-img}
   Fonte: \cite[p. 1]{iconixuml}
\end{figure}

A primeira atividade, Análise de Requisitos, é a fase que se envolve o início do projeto. A primeira tarefa é a conversa com o cliente, público alvo do projeto e todos interessados do projeto. Essa conversa resultará em inúmeros requisitos que darão uma estimativa de como deve ser a estruturação do projeto, visto que nesta fase, o projeto ainda não está estruturado.

O passo seguinte da análise de requisitos é entender o domínio do problema. Essa tarefa consiste em garantir que não hajam problemas de ambiguidade na compreensão do problema. O modelo de domínio (geralmente representado em \emph{UML -- Unified Modeling Language}) é usado para que fique clara a comunicação entre os membros do projeto.

Por fim, os Requisitos Comportamentais baseiam-se em cenários de como será a interação do usuário com o sistema. Começar com uma \emph{GUI} (Interface Gráfica do Usuário -- \emph{Graphical User Interface}) é a forma de apresentar o contexto do usuário que está sendo modelado \cite{iconixuml}.

A segunda atividade, Análise Preliminar, é uma etapa dentre a análise e o design. O diagrama de robustez descreve exatamente a atividade. O diagrama permite verificar as falhas que existem e identificar sugestões para que seu projeto não tenha possíveis ambiguidades, refinando os modelos de caso de uso da atividade anterior.

A próxima atividade, Projeto, é a construção do sistema, finalizando a visão dinâmica do sistema e início da visão estática. A preocupação no momento é a eficiência do sistema. Os principais modelos anexados a essa etapa são: Diagrama de Sequência, que fornece detalhes do que será implementado; Refinamento do Modelo de Domínio, onde devem ser inseridas as operações aos objetos de domínio e resolver os problemas reais, como o padrão de projeto que o sistema deve se adequar.

Por fim, a etapa de Implementação, é a atividade de codificação e testes com base no diagrama de Classes, formado a partir do Modelo de Domínio e Diagrama de Sequência, contendo as interações com sistemas externos.



\section{Trabalhos Relacionados}
\label{relacionados}
Nesta seção serão apresentados trabalhos acadêmicos que são relacionados com os fundamentos para o desenvolvimento desta pesquisa. Os trabalhos relacionados apresentam diferentes técnicas para solucionar o roteamento de veículos.

\subsection{Uma Metodologia para Roteamento de Veículos Escolares Utilizando Sistemas de Informação Geográfica \cite{maxwell}} 

Esta dissertação de mestrado pauta-se no problema de roteamento de veículos de colégios da rede estadual na região rural do Rio de Janeiro, abordando o planejamento das rotas de uma frota de veículos que transporta alunos a partir da origem até as respectivas escolas. O problema de roteamento de ônibus escolares -- \emph{School Bus Routing Problem (SBRP)} -- é derivado do VRP. E assim como o VRP, o SBRP é um problema \emph{NP-Difícil}. Portanto, as rotas foram obtidas através de um método meta-heurístico denominado \emph{Adaptative Large Neighborhood Search} (ALNS) utilizando a ferramenta \emph{VRP Spreadsheet Solver}.

Rosa \citeyear{maxwell} apresentou a metodologia para a resolução do problema em oito fases:
\begin{enumerate}
    \item \textbf{Definir abrangência}: a abrangência pode ser definida tanto geograficamente, ou seja, delimitar os municípios, distritos ou até colégios que serão objetivo da pesquisa, quanto os turnos que serão abordados.
    
    \item \textbf{Geocodificar endereço das escolas, origem dos alunos e pontos de embarque}: trata-se de uma técnica de converter endereços em coordenadas geográficas -- representado através de um par coordenado latitude, longitude. A geocodificação foi realizada através do \emph{Google Maps Geocoding API}.
    
    \item \textbf{Definir as características}: devido à delimitação de problemas reais, existem diversas classes dentro do problema do roteamento de veículos escolares que optam por abordá-los ou ignorá-los. Por exemplo, características específicas de um ambiente urbano ou rural, carregamento misto entre escolas, transporte de alunos especiais, tipo de frota entre outros. O trabalho abrangeu 15788 alunos do período noturno de 173 escolas localizadas em áreas rurais. O transporte realizado por frota heterogênea onde não é permitido carregamento misto. Visando diminuir o tempo ou distância total de viagem do veículo e levando em consideração a capacidade do veículo e o tempo máximo de percurso.
    
    \item \textbf{Calcular a distância e tempo de percurso}: nesta fase, a distância e o tempo de percurso são calculados e inseridos em uma matriz de Origem e Destino (OD). Para estimativa do tempo dos percursos foram consideradas distâncias Euclidiana e Geodésica divididas pela velocidade média desejada. Já a distância real, que leva em consideração as vias foi obtida através do Google Maps.
    
    \item \textbf{Montar banco de dados de georreferenciamento e da frota de veículos}: foi realizada a estruturação dos dados de localização, distância e dos veículos baseados na \emph{VRP Spreadsheet Solver}
    
    \item \textbf{Aplicar ferramenta para obter as rotas dos veículos}: foi aplicado uma meta-heurística denominada \emph{Adaptive Large Neighborhood Search} utilizando o \emph{VRP Spreadsheet Solver} para obtenção das rotas de cada veículo.    
    
    \item \textbf{Geoespacilizar as rotas}: trata-se da visualização das rotas criadas através de uma interface gráfica. No trabalho foi utilizado o Google Maps como exemplo.
    
    \item \textbf{Elaborar diagnóstico}: etapa fundamental para verificar a metodologia proposta. No caso do transporte de estudantes rurais, o diagnóstico é realizado através de parâmetros pré-estabelecidos que servem de comparação para o planejamento de ações pontuais ou globais de forma a melhorar o desempenho do sistema. De acordo com o Fundo Nacional do Desenvolvimento da Educação (FNDE), um planejamento deve considerar elementos físicos (vias, veículos), lógicos (legislação, estrutura normativa e de gestão) e os agentes (motorista e responsáveis). 
\end{enumerate}

Assim como no presente trabalho, o problema da dissertação também pertence à classe \emph{OVRP}, embora esteja contida especificamente na classe de problema de roteamento de veículos escolares (\emph{SBRP}). Outra semelhança é no modelo de planejamento das rotas que são fixas, ou seja, não são modificadas em tempo real. Caracterizando uma diferença entre os trabalhos, o problema é apresentado no contexto de locomoção rural do interior do Rio de Janeiro enquanto este projeto tem como foco a região metropolitana de São Paulo.


\subsection{A survey of models and algorithms for optimizing shared mobility \cite{MOURAD2019323}}

O artigo feito por \cite{MOURAD2019323} é uma referência importante para a resolução do VRP em que os autores abordam exatamente a questão do problema emergente do aumento de número de veículos nas grandes cidades. O artigo traz resoluções para o problema de VRP em 2 aspectos: pré-organizado e em tempo-real. Pré-organizado é quando se tem informações  registradas previamente, ou seja, as informações de origem e destino são inseridas antes do cálculo da rota. Já em tempo-real, todas as variáveis de cálculo são coletadas em tempo de execução, neste contexto, tornando possível calcular mudanças repentinas na rota, como fechamento de vias, ou trânsito intenso em certas vias por exemplo. O artigo também mostra que a maioria das variantes do VRP trata de viagens com múltiplos passageiros, uma vez que um veículo que suporta mais pessoas traz maior diminuição de veículos nas ruas.
A partir da variante dos problemas, se tem, também, uma variedade de restrições que são consideradas para cada caso do VRP, dentre elas tem se:

\begin{itemize}
\item Restrições de Roteamento -- Onde os veículos podem ou não passar.
\item Restrições de Tempo -- Tempo máximo de viagem e janelas de tempo
\item Restrições de Capacidade -- Capacidade total do veículo
\item Restrições de Custo -- Custo total da viagem para os passageiros
\item Restrições de Sincronia -- Atendimento de demandas em tempo real
\end{itemize}

 Além de avaliar o VRP em diferentes variantes do problema (diferentes veículos e demandas), há uma comparação bem ampla de resultados entre métodos heurísticos e exatos aplicados em simulação. Na delimitação deste trabalho, foi escolhido uma resolução pré-organizada ao invés de trabalhar com informações em tempo real. Assim, é uma fonte comparativa para que seja possível entender qual método se aproxima mais do ideal para o contexto do protótipo. O projeto utiliza de restrições que estão dentro da categoria de informações pré-organizadas, nós utilizaremos as seguintes restrições para alcançar objetivo:

\begin{itemize}
\item Restrições de Roteamento
\item Restrições de Tempo
\item Restrições de Capacidade
\item Restrições de Custo Monetário
\end{itemize}

Essas restrições são o suficiente para satisfazer uma viagem pré-organizada dos usuários do transporte coletivo, se a rota atender todas essas restrições, ela será dada como uma boa rota alternativa para o usuário que deseja mais conforto e praticidade na locomoção diária.

 
 \subsection{Algoritmos de Solução para o Problema do Caixeiro Viajante com Passageiros e Quota \cite{silva2017algoritmos}}
 
A dissertação de mestrado de Jean Gleisson apresenta Algoritmos de Solução para o Problema do Caixeiro Viajante com Passageiros e Quota (\emph{PCVPQ}), sendo uma derivação no problema de roteamento de veículo. A dissertação tem como objetivo reduzir o custo total da viagem compartilhando o custo entre os passageiros que solicitam a carona, porém com o enfoque na soma mínima de quotas coletadas nas cidades visitadas pelo caixeiro. A quota é o valor do custo monetário de deslocamento entre as cidades que será dividida igualmente entre os ocupantes, inclusive o motorista.
Existem algumas restrições que devem ser respeitadas, como: O motorista é livre para decidir quais cidades irá visitar; a quota da viagem não pode ultrapassar o valor máximo fixado pelo próprio passageiro e não é levada em consideração a duração da viagem, tempo que permanecerá nas localidades ou o número de cidades que serão visitadas durante a viagem e não haverá uma coleta de bônus na cidade de origem do caixeiro.
Na proposta apresentada nesta dissertação, foi considerado o grafo $G = (N, M, B)$, onde:

$N$ é o conjunto de vértices que representam as cidades;
$M$ é o conjunto de arestas que representam as rodovias que ligam as cidades, onde também há um custo associado à aresta cujo valor é o custo de deslocamento entre as cidades;
$B$ é o bônus associado a cada cidade, que será coletado no momento que ocorre a visita da cidade.

Para a solução, o autor estudou os seguintes métodos:
A heurística de carregamento ótimo (\emph{NAIVE}) visa obter a rota de menor custo possível, e para essa rota, obter o embarque ótimo para o carro.
A metaheurística \emph{GRASP} para o \emph{PCVPQ} descrito consiste na utilização de uma roleta para selecionar os vértices de acordo com o seu bônus e os incorporando à rota.
As variações de algoritmos meméticos e genéticos com mutação foram estudadas com o objetivo de que as mutações sejam aplicadas para minimizar a quota durante a execução da rota.

Como conclusão, o autor encerra dando a ênfase no problema, pois tem potencial para ser aplicado em sistemas de otimização de transporte colaborativo e os melhores desempenhos qualitativos foram alcançados pelos algoritmos \emph{GRASP}, \emph{MEM-SPLIT} e \emph{MEM-SCX}, destacando-se o \emph{GRASP} com o melhor tempo.

Assim como no trabalho mencionado, temos como objetivo o transporte coletivo de passageiros, porém levaremos em consideração:
\begin{itemize}
    \item As cidades visitadas pelo motorista, para que haja compatibilidade de passageiros da mesma região;
    \item Atender um grande número de pessoas num mesmo veículo e reduzir o tempo da viagem dos passageiros; 
    \item O tempo de viagem total, assim teremos como uma meta a viagem ser confortável para o passageiro.
\end{itemize}

Embora os objetivos e alguns métodos de resolução do problema sejam parecidos, as restrições serão diferentes, por exemplo, não teremos o sistema de coleta do bônus descrito, pois o mesmo não é relevante na nossa solução. Portanto, é importante entender que as delimitações podem gerar muitas soluções para um mesmo tipo de problema, de acordo com cada segmentação levada em consideração em cada projeto.

